\documentclass[twocolumn,aps,prd,reprint]{revtex4-1}
\usepackage{blindtext}
%\usepackage[justification=centering]{caption}
\usepackage{graphics,setspace,enumitem,graphicx,textpos,placeins,float}
\begin{document}
\bibliographystyle{plain}
\title{Electroweak Supersymmetry Simplified Models to Explain Razor - Hγγ Excess}
\author{Nicholas Bower}
\email{Nicholas\_Bower@brown.edu}
\author{Javier Duarte}
\author{Cristian Pena}
\author{Maria Spiropulu}
\affiliation{California Institute of Technology}
%\begin{spacing}{1.2}
\begin{abstract}
The Compact Muon Solenoid (CMS) collaboration has reported an excess of events consistent with a Higgs boson decaying into a pair of photons and in association with jets. This excess of events over the Standard Model prediction was observed in proton-proton collision at a center-of-mass energy of 8 TeV, and could be hint to observe beyond the SM  (BSM) physics. The data analysis carried out by CMS uses the specialized razor variable which are sensitive to BSM physics and particularly to Supersymmetry (SUSY). In particular electroweak SUSY is a likely candidate to explain the observed excess and could help understanding its nature. We present studies that simulates bottom squarks pair production that lead to BSM production of higgs bosons and could decay to a pair photons or b-quarks in proton-proton collisions at a center-of-mass energy of 13 TeV, we also study the feasibility to observed this model in an alternative final state where the identified higgs candidate is reconstructed from two b-tagged jets.
\end{abstract}
\maketitle
\section{Introduction}
The Compact Muon Soleniod Team (CMS) at cern is focused on the some of the most pressing issues facing all of phycs. of primary interest is the understanding of the standard model and how many inexplicable phenomena change our view of this model. SM is incomplete, amongst many other failuires it directly fails to describe dark matter, and modern theory is rapidly expanding predicting a great many particles that are not included in the standard model. All of this behavior is described by the umbrella term Beyond the Standard Model(BSM.)

The two BSM phenomena that hold the largest profile are also of pronciple interest of the Cal tech group at CMS: Supersymmetry(SUSY) and Dark Matter(DM). So far Dark matter has only directly been observed on the outskirts of galaxies adding incredible amounts of mass to these galaxies. Dark matter has been observed to comprise 26\% of all matter in the universe roughly 5 times that of the SM matter. The leading theory is that these particles could be interacting with SM particles by the weak force, earning them the designation of weakly interacting massive particles (WIMP.) If this is the case then DM should be observable in a suffieciently high energy particle collider. 

Along with Dark Matter SUSY is one of the biggest challenges facing the standard model. SUSY demonstrates a new way to look at particle interactions. In the simplest terms it is the existence of an operator capable of changing a fermion to a boson and vice versa. This is an extremely important theory. It predicts partner particles for all existing particles. It completely shapes the picture of how we view high energy physics. There must exist all new mediators, new force carriers with totally different behaviors because of their new classification. SUSY has long been theorized but it has yet to be directly observed. The SUSY states have unknown mass so we remain unable to fully classify these relation ships and we cannot truly proceed with the science until we verify SUSY. as a result of this SUSY detection and study is of principal interest to the high energy community.

BSM phenomena are difficult to measure. DM interacts very rarely with matter, and is not directly measurable with current detectors. It must also be first creatted in a detector which is nontrivial. In the case of SUSY states, this creatign is the challenging part, and detectors are not deigned to detect these states as well. To find BSM particles the typical method is to create them via a monojet collision. the resultant products are measured and the misign TRansverse energy is calculated. 
\subsection{Higgs To Gamma Gamma}
 A experiment done colliding two quarks to produce higgs bosons. When compared to the simulations of SM phenomena there was found to be an abundance of particles with MR 150-300GeV. The specifics of what this actually means will be discussed later but essentially this meant there was some BSM behavior for particles at this energy. It is possible that this is evidence of the generation of a SUSY state, and dark matter candidates, but as of yet the data does not quite fit the theory developed to attempt to describe it. Another (somewhat more likely) possibility is that this is evidence of a new interaction between particles that not possible in the SM.

This experiment was repeated at 13TeV, and the same result was found at this higher energy. This lends credibility to the previous data, and showed that it was worthy of significant investigation. As of now this phenomena has only been observed in the case where the Higgs boson created by the initial collision decays into two photons, but it also must be observable when the higgs decays into two b-jets. Experimentally speaking this must be the next step. We must continue to verify that this data is valid and gather the tools necessary to further develop the theory surrounding it. It is not a forgone conclusion that we will find this effect in the Hbb data. bb data is typically very noisy, which would make this abundance harder to differentiate, although the H to bb decay pattern is more common than the photon decay, leading to more usable data. 
\section{Level 1}
\subsection{Methods}
Several variations of hadronic L1 triggers were investigated. We looked at a few forms of the $H_T$ trigger. At the L1 level, $H_T$ is calculated as the sum of the $E_T$s of the 4x4 trigger towers \cite{ucsb}.  These regions are each defined in azimuthal by $20^{\circ}$ in the azimuthal angle $\phi$, with divisions in psuedorapidity $\eta$ as well \cite{hlt}. To investigate the relationship between the razor variables and this trigger, two dimensional binned efficiency plots in generator-level $M_{R}$ and $R^2$ were created with different selections on L1 data. In general, the baseline selection included a calorimeter region transverse energy $E_T$ threshold of 7 GeV, with an $|\eta| <$  3, or a jet $p_T$ minimum of 10 GeV if jet selection was involved. For single jet studies, we only had a baseline jet minimum. The datasets for which these efficiencies were calculated included $t\bar{t}$+jets, which is a common Standard Model background for SUSY searches, as well as the SUSY signals T2tt and T2qq. T2qq refers to first and second generation squark-antisquark production, decaying to quark/antiquark and two LSPs. T2tt refers to stop/antistop pair production, decaying to top/antitop and two LSPs. The T2tt sample had $m_{\tilde{t}}=800$ GeV and $m_{\tilde{\chi}^0_1}=100$ GeV, and the T2qq samples had $m_{\tilde{q}}=400$ GeV and $m_{\tilde{\chi}^0_1}=150$ GeV, $m_{\tilde{q}}=450$ GeV and $m_{\tilde{\chi}^0_1}=400$ GeV, and $m_{\tilde{q}}=600$ GeV and $m_{\tilde{\chi}^0_1}=100$ GeV. We also investigated a T1qqqq sample for our quadjet studies and our jet algorithm studies, which refers to gluino pair production decaying to four quarks and two LSPs. A summary of the pair-produced particles and their decay products can be found in Table \ref{table:signals}. The SUSY signal samples were generated by Maria D'Alfonso from UCSB \cite{ucsb}. 
\subsection{Results}
\par $H_T$ cuts over a broad range of energies were examined. In figure 5, the efficiency over a region of $M_R$ and $R_2$ from 400 to 1450 GeV in $M_R$ and 0.25 to 1.5 in $R^2$ was plotted for increasing cuts in GeV. We see here that all samples drop to efficiency below 0.95 after the $H_T$ selection is greater than 210-220 GeV. We aim to keep efficiencies above 0.95 because low efficiencies pose a problem for fits later on in the razor analyses \cite{talk}. If we look lower in $M_R$ and $R^2$, we can see that the efficiency drops for low $M_R$ (Fig. \ref{fig:eff_t2tt}-\ref{fig:eff_t2com}).  
\par The addition of a $\Delta \phi$ cut was also examined for 2.6 radians. This is motivated by the trend that quantum chromodynamics (QCD) background mostly lies at high $\Delta \phi$, where $\Delta \phi$ is the azimuthal angle difference between any two jets. To minimize the rate of this new trigger while selecting for events lower in $H_T$, members of the CMS internal SUSY Group proposed this cut with the added criteria of a leading jet $p_T$ cut of 120 GeV as well as a subleading jet $p_T$ cut of 30 GeV \cite{susy_group}. With the addition of this trigger, we see a marked improvement in lower $M_R$ (Fig. \ref{fig:eff_t2com}, \ref{fig:comp}).
\par An alternate trigger was proposed by the CMS internal SUSY Group that involves selection on $H_T^{miss}$ values, which is defined as the negative vectorial sum of the region $E_T$ values \cite{susy_group}. The trigger requires $H_T > 130$ GeV in conjunction with $H_T^{miss}/H_T >$ 0.3 to keep rates low. This improves on the previously mentioned trigger with the angle cut (Fig. \ref{fig:at}, \ref{fig:ratio_at}). To fully evaluate the usefulness of these two proposed triggers, events passing certain regions in the $M_R$,$R^2$ plane were directly compared (Fig. \ref{fig:region}). The $H_T^{miss}$ has a noticeable improvement in region I of low $M_R$, high $R^2$, an area which is particularly important for dark matter searches \cite{dark_razor}.
\begin{figure}[h]
 %\centering
 \includegraphics[width=0.45\textwidth]{cones_T1_1400_100_0,4.pdf}
\caption{\label{fig:t1_clstr} $M_R$ distributions of T1 for anti-$k_T$ inclusive jets, which are then clustered with either the megajet algorithm or the $k_T$ algorithm, using various cone sizes. The legend numbers specify the cone size.}
\end{figure}
\begin{figure}[h]
 %\centering
 \includegraphics[width=0.45\textwidth]{r2_cones_T1_1400_100_0,4.pdf}
\caption{\label{fig:t1_clstr_r2} $R^2$ distributions of T1 for anti-$k_T$ inclusive jets, which are then clustered with either the megajet algorithm or the $k_T$ algorithm, using various cone sizes. The legend numbers specify the cone size. }
\end{figure}
\begin{figure}

\bibliographystyle{apsrev4-1} % Tell bibtex which bibliography style to use
\bibliography{progress_report} 
\end{document}